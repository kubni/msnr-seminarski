

 % !TEX encoding = UTF-8 Unicode

\documentclass[a4paper]{report}

\usepackage[T2A]{fontenc} % enable Cyrillic fonts
\usepackage[utf8x,utf8]{inputenc} % make weird characters work
\usepackage[serbian]{babel}
%\usepackage[english,serbianc]{babel}
\usepackage{amssymb}

\usepackage{color}
\usepackage{url}
\usepackage[unicode]{hyperref}
\hypersetup{colorlinks,citecolor=green,filecolor=green,linkcolor=blue,urlcolor=blue}

\newcommand{\odgovor}[1]{\textcolor{blue}{#1}}

\begin{document}

\title{Šta čini univerzitetski kurs teškim?\\ \small{Jovan Vukićević, Jovan Škorić, Nikola Kuburović, Radenko Nikolić}}

\maketitle

\tableofcontents

\chapter{Uputstva}
\emph{Prilikom predavanja odgovora na recenziju, obrišite ovo poglavlje.}

Neophodno je odgovoriti na sve zamerke koje su navedene u okviru recenzija. Svaki odgovor pišete u okviru okruženja \verb"\odgovor", \odgovor{kako bi vaši odgovori bili lakše uočljivi.} 
\begin{enumerate}

\item Odgovor treba da sadrži na koji način ste izmenili rad da bi adresirali problem koji je recenzent naveo. Na primer, to može biti neka dodata rečenica ili dodat pasus. Ukoliko je u pitanju kraći tekst onda ga možete navesti direktno u ovom dokumentu, ukoliko je u pitanju duži tekst, onda navedete samo na kojoj strani i gde tačno se taj novi tekst nalazi. Ukoliko je izmenjeno ime nekog poglavlja, navedite na koji način je izmenjeno, i slično, u zavisnosti od izmena koje ste napravili. 

\item Ukoliko ništa niste izmenili povodom neke zamerke, detaljno obrazložite zašto zahtev recenzenta nije uvažen.

\item Ukoliko ste napravili i neke izmene koje recenzenti nisu tražili, njih navedite u poslednjem poglavlju tj u poglavlju Dodatne izmene.
\end{enumerate}

Za svakog recenzenta dodajte ocenu od 1 do 5 koja označava koliko vam je recenzija bila korisna, odnosno koliko vam je pomogla da unapredite rad. Ocena 1 označava da vam recenzija nije bila korisna, ocena 5 označava da vam je recenzija bila veoma korisna. 

NAPOMENA: Recenzije ce biti ocenjene nezavisno od vaših ocena. Na osnovu recenzije ja znam da li je ona korisna ili ne, pa na taj način vama idu negativni poeni ukoliko kažete da je korisno nešto što nije korisno. Vašim kolegama šteti da kažete da im je recenzija korisna jer će misliti da su je dobro uradili, iako to zapravo nisu. Isto važi i na drugu stranu, tj nemojte reći da nije korisno ono što jeste korisno. Prema tome, trudite se da budete objektivni. 
\chapter{Recenzent \odgovor{--- ocena: 3} }


\section{O čemu rad govori?}
% Напишете један кратак пасус у којим ћете својим речима препричати суштину рада (и тиме показати да сте рад пажљиво прочитали и разумели). Обим од 200 до 400 карактера.
Može se reći da su studentski kursevi jedna od najvećih, ako ne i najveća prepreka studentu tokom njegovog studiranja. Rad je imao za cilj da istraži ključne kriterijume po kome se može reći da je neki kurs težak, i time se uočiti obrazac po kome oni dobijaju tu etiketu. Nedostupnost kvalitetnog materijala najveći je problem studentima matematičkog fakulteta.

\section{Krupne primedbe i sugestije}
% Напишете своја запажања и конструктивне идеје шта у раду недостаје и шта би требало да се промени-измени-дода-одузме да би рад био квалитетнији.
Glavna stvar koja mi je nedostajala u radu jeste razrada i dublje objašnjenje ankete. Iako rezultati postoje, od celog rada, oni uzimaju oko 20\%. Čitaoci nisu upućeni o raznim informacijama kao što su:
\begin{enumerate}
\item Ko je radio ovu anketu?
\odgovor{Videti odgovor za primedbu 5.}
\item Koliko ljudi je radilo anketu?
\odgovor{Videti odgovor za primedbu 5.}
\item Koje informacije znamo o studentima (godina studiranja itd. ako takvi podaci postoje)?
\odgovor{Videti odgovor za primedbu 5.}
\item Da li postoje različiti segmenti ankete, ako da, koja su pitanja (barem ona glavna)?
\odgovor{ Postoje, ali smo smatrali da je nepotrebno dodavanje segmenata ankete jer
 su oblasti na koje je rad podeljen ujedno i segmenti ankete.}
\item Da li su pitanja vezana striktno za matematički fakultet ili ima i opštih pitanja?
\odgovor{Dodat je tekst: “U odgovaranju na pitanja iz ankete učestvovao je 61 student
Matematičkog fakulteta u Beogradu, sa različitih stepena studija (slika 3):” i njemu
odgovarajuća slika : “Slika 3:  Raspodela studenata prema stepenu studija”}
\end{enumerate}
Takođe, nema grafikona kao što je bar plot i pie plot koji bi vizuelno mogli lepo da prikažu nezodovoljstvo studenata.
\odgovor{ Dodat je bar plot: “Slika 4: Uticaj nedostatka interakcije predavača sa
 studentima”, koji prikazuje jedan od glavnih faktora koji utiče na težinu kursa.}
 Anketa je rekao bih dobro urađena (u smislu opisivanja rezultata), ali smatram da treba da se proširi sa gorenavedenim predlozima.
\newline

Druga veća sugestija bi bila da se doda diskusija i poređenje između postojeće literature i vaših rezultata.
\odgovor{ Nije dodato poređenje jer bi ceo rad bio izvan granica od 12 strana.}
Koja mišljenja se poklapaju? Da li postoji nekih anomalija između popularnih istraživanja i mišljenja studenata?
\newline

Za kraj, dodao bih da bi sažetak trebao da sadrži i rezultate, a ne samo opis problema. On mora enkapsulirati ceo rad.
\odgovor{ Nismo dodali rezultate u sažetak, jer smo pokušavali da ga održimo konciznim i da
 rad bude unutar granica od 12 strana.}

\section{Sitne primedbe}
% Напишете своја запажања на тему штампарских-стилских-језичких грешки
S obzirom da su zamerke sitnog karaktera, staviću ih po tezama:
\begin{enumerate}
\item U sekciji 2 (Organizacija nastave) imate rečenicu koja sadrži: na osnovu dostupne literature..., a zapravo je pasus odrađen bez reference. Stavite reference iza rečenice.
\odgovor{Rečenica je deo uvoda onoga o čemu će biti cela sekcija, zato nije dodata referenca
 na literaturu jer se smatra na literaturu koja sledi u nastavku teksta.}
\item Reč kurikulum preimenovati u kurs radi konzistentnosti.
\odgovor{Izmenjeno.}
\item Boldovati ključne brojke u tabeli i time ukazati na najveći problem.
\odgovor{Boldovani najveći procenti u svakom redu tabele.}
\item Dodati email-ove u naslovu rada.
\odgovor{Dodati mejlovi svih autora.}
\end{enumerate}

\section{Provera sadržajnosti i forme seminarskog rada}
% Oдговорите на следећа питања --- уз сваки одговор дати и образложење

\begin{enumerate}
\item Da li rad dobro odgovara na zadatu temu?\\
Rad, iako sadrži navedene mane po mom mišljenju, dobro odgovara na temu. Navedeno je puno referenci, jako dobro je objašnjen problem. Sekcija za rezultate i analizu rezultata postoji. Iako treba razraditi sekciju oko ankete, ona je u suštini dobra.
\item Da li je nešto važno propušteno?\\
Smatram da detaljnija analiza rezultata (dodavanje informacija, grafikona i poređenje sa postojećom literaturom) je ključna da ovaj rad postane odličan. 
\item Da li ima suštinskih grešaka i propusta?\\
Rad je dobar. Posle razrade ankete smatram da neće postojati velikih mana. 
\item Da li je naslov rada dobro izabran?\\
Jeste, rad je dobro pokrio i referencirao temu. Anketa je jako korelisana sa naslovom rada.
\item Da li sažetak sadrži prave podatke o radu?\\
To je jedna od glavnih napisanih sugestija. Sažetak mora da sadrži rezultate ankete, a ne samo opis postojećeg problema. 
\item Da li je rad lak-težak za čitanje?\\
Rad je rekao bih lak za čitanje, nisam imao poteškoće u razumevanju problema i njihovih uzroka.
\item Da li je za razumevanje teksta potrebno predznanje i u kolikoj meri?\\
Rad je dobro referenciran i jednostavno objašnjen. Mogao sam vrlo lako kroz jednu-dve iteracije da zaključim sve o radu.
\item Da li je u radu navedena odgovarajuća literatura?\\
Jeste, to je možda najbolja stvar u ovom radu. Pokrivenost referencama i literaturom. Autori su odradili odličan posao što se tiče objašnjavanja problema.
\item Da li su u radu reference korektno navedene?\\
Jesu, to se može i videti kroz činjenicu da je većina rada zapravo objašnjavanje nalaza iz literature.
\item Da li je struktura rada adekvatna?\\
Rekao bih da da, posle rada na zamerkama oko ankete. Diskusija i barem par pasusa o samoj anketi pre pokazivanja rezultata su bitni u rezumevanju rada.
\item Da li rad sadrži sve elemente propisane uslovom seminarskog rada (slike, tabele, broj strana...)?\\
Iako sadrži slike vezane za literaturu i problem, rad nema slike iz ankete. Ostalo je koliko primećujem zadovoljeno.
\item Da li su slike i tabele funkcionalne i adekvatne?\\
Slike koje opisuju anketu ne postoje. Što se tiče tabele, sugerisao sam kao sitnu zamerku da treba pojačati vidljivost bitnih rezultata.
\end{enumerate}

\section{Ocenite sebe}
% Napišite koliko ste upućeni u oblast koju recenzirate: 
% a) ekspert u datoj oblasti
% b) veoma upućeni u oblast
% c) srednje upućeni
% d) malo upućeni 
% e) skoro neupućeni
% f) potpuno neupućeni
% Obrazložite svoju odluku
S obzirom da sam i ja student (master student matematičkog fakulteta) smatram da sam srednje upućen u oblast. Nikada nisam istraživao literaturu, ali sam imao iskustva i imam svoja mišljenja šta čini neki univerzitetski kurs teškim.

\chapter{Recenzent \odgovor{--- ocena: 4} }


\section{O čemu rad govori?}

Autori rada analiziraju šta čini univerzitetski kurs teškim, baveći se prevashodno organizacijom ispita i nastave. U razradi su za obe teme posmatrani različiti aspekti i njihov uticaj na doživljaj težine kursa. Na kraju, analiziraju se rezultati sprovedene ankete i izvode zaključci na osnovu njih.

\section{Krupne primedbe i sugestije}

Dobro odradjeno:

\begin{itemize}
    \item Donošenje zaključaka kombinovanjem više različitih izvora. Takodje, zaključci su dobro argumentovani.

    \item Dobro razradjene obe podteme koje su dikutovane u razradi.

    \item Temeljno obradjena analiza rezultata ankete.

\end{itemize}


Primedbe i sugestije:

\begin{itemize}
    \item Mejlovi da se dodaju za sve autore (1. strana).
    \odgovor{Dodati su mejlovi.}
    \item 3.1 Onlajn nastava i njeni izazovi, nalazi se u okviru organizacije ispita, deluje da se više bavi organizacijom nastave nego ispita. Takodje, uporedjivanjem sa ostalim podnaslovima u ovom poglavlju ova razlika postaje još očiglednija. Razmotriti prebacivanje u poglavlje 2.
    \odgovor{Posle ponovnog čitanja rada, prebačeno je u glavu 2 kao poslednja stavka.}
    \item Rečenica nije paragraf (3.1, 3.4). U 3.1 recimo ako se već razradjuje nesigurnost studenata, da se razmotri zašto je to tako (recimo smanjena razmena iskustava sa ostalim kolegama i sl). Za 3.4, može da se spoji sve u jedan pasus, jer mu idejno pripada kao neka vrsta zaključka.
    \odgovor{U sekciji 3.1, sada 2.4, dodatno je razjašnjeno u kom pogledu je studentima nastala nesigurnost, dok u sekciji 3.4, sada 3.3, spojeno sve u jedan pasus.}
    \item U tabeli da se objasni šta predstavljaju ocene od 1 do 5 (može da se ubaci recimo u caption). Takodje bi trebalo navesti da se u tabeli nalaze procenti. Naslov tabele ide iznad.
    \odgovor{Dodato šta predstavljaju ocene 1 i 5, dodati procenti, naslov tabele stavljen iznad.}
    \item U delu 3.3 "Istraživanja ukazuju da ovakva razlika..."\hspace{0.1cm} nedostaje referenca.
    \item Rad bi bio pregledniji ukoliko bi sadržao grafikone, bar za ključna pitanja iz dela za anketu (dijagrami raspodele odgovora).
    \odgovor{Dodata 2 grafikona u deo sa rezultatima.}
\end{itemize}

\section{Sitne primedbe}


\begin{itemize}
    \item Izmeniti "uključuju uključivanje", 2.1 treći pasus.
    \odgovor{Zamenjeno sa "uključuju dodavanje".}
    \item Nedostaje tačka, 2.2 ispred "Još
    jedan važan aspekt...".
    \odgovor{Dodata tačka.}
    \item "(McDowell i saradnici, [8])"\hspace{0.1cm} dovoljna je samo numerička referenca, slično  "(Gamage i saradnici, [4])".
    \odgovor{Ostavljene samo reference.}
    \item Greške u kucanju naslova "Ogranizacija".
    \odgovor{Ispravljena slovna greška.}
    \item Promeniti "Usmena vs. pismena provera znanja"\hspace{0.1cm} da bude formalnije, tj. izbaciti vs.
    \odgovor{Promenjeno u "Odnos usmene i pismene promene znanja".}
\end{itemize}


\section{Provera sadržajnosti i forme seminarskog rada}


\begin{enumerate}
\item Da li rad dobro odgovara na zadatu temu?\\
Da, rad je pokrio većinu ključnih aspekata i dobro ih obrazložio. 
\item Da li je nešto važno propušteno?\\
Po rezultatima iz ankete, a koje možemo videti u tabeli. Studenti naglašavaju da je nedostupnost kvalitetnog materijala najbitniji aspekat (72\% je dalo ocenu 5), a kome nije posvećena posebna pažnja. 
\item Da li ima suštinskih grešaka i propusta?\\
Zaključci u radu su dobri, pored već navedenih propusta nema dodatnih zamerki na suštinu rada.
\item Da li je naslov rada dobro izabran?\\
Da, naslov je dobro izabran.
\item Da li sažetak sadrži prave podatke o radu?\\
Da, sažetak sadrži tačne i relevantne podatke o radu, pružajući jasan pregled njegovog sadržaja i ciljeva.
\item Da li je rad lak-težak za čitanje?\\
Može se reći da je lak. Korišćen je dosledan i jasan stil pisanja.
\item Da li je za razumevanje teksta potrebno predznanje i u kolikoj meri?\\
Za razumevanje teksta nije potrebno veliko predznanje. Dovoljno je elementarno poznavanje kako fakultet funkcioniše.
\item Da li je u radu navedena odgovarajuća literatura?\\
Da.
\item Da li su u radu reference korektno navedene?\\
Da, uz male tehničke greške pri navodjenju.
\item Da li je struktura rada adekvatna?\\
Osim već navedenog u delu za primedbe, ostalo je u redu.
\item Da li rad sadrži sve elemente propisane uslovom seminarskog rada (slike, tabele, broj strana...)?\\
Da, ispunjava sve uslove.
\item Da li su slike i tabele funkcionalne i adekvatne?\\
Navedeni već sitni propusti, slike su adekvatne i funkcionalne.
\end{enumerate}

\section{Ocenite sebe}

Pošto sam student i svakodnevno diskutujem sa ostalim kolegama uslove i kvalitet studiranja, smatram da sam upućen u ovu oblast.


\chapter{Recenzent \odgovor{--- ocena: 2} }


\section{O čemu rad govori?}
% Напишете један кратак пасус у којим ћете својим речима препричати суштину рада (и тиме показати да сте рад пажљиво прочитали и разумели). Обим од 200 до 400 карактера.
Rad govori o okolnostima koje čine univerzitetske kurseve teškim, kakve okolnosti bi bile idealne da univerzitetski kursevi ne budu teški i posledice koje teški univerzitetski kursevi ostavljaju na studente.

\section{Krupne primedbe i sugestije}
% Напишете своја запажања и конструктивне идеје шта у раду недостаје и шта би требало да се промени-измени-дода-одузме да би рад био квалитетнији.
Smatram da je trebalo pomenuti da previše olakšavanja univerzitetskih kurseva može negativno uticati na studente. Na primer, ukoliko će gradivo na predavanjima i vežbama biti previše slično zadacima na ispitu, studenti mogu da se pripreme za ispit bez preteranog razumevanja gradiva.
\odgovor{Dodata je rečenica ``Treba uzeti u obzir da prevelika
sličnost izmedu zadataka s predavanja i zadataka na ispitu može podstaći
studente da nauče ispitno gradivo napamet, bez dubljeg razumevanja`` na kraju poglavlja 3.3.}

\section{Sitne primedbe}
% Напишете своја запажања на тему штампарских-стилских-језичких грешки
Fale email-ovi trojice kolega. \odgovor{Dodati su emailovi.} Fali tačka na kraju rečenice u odeljku 2.2. \odgovor{Dodata je nedostajuća tačka.} U odeljku 3.1 postoji greška u kucanju (neodostatak umesto nedostatak). \odgovor{Ispravljena je greška.} U odeljku 4 je korišćena ijekavica (uvijek) iako se svuda koristi ekavica. \odgovor{Tekst je prepravljen da bude konzistentan sa ekavicom.} Takođe, u odeljku 4 postoji greška u kucanju (ne umesto je). \odgovor{``ne`` je prepravljeno u ``nije``, a ne ``je``, radi očuvavanja smisla rečenice.} U tabeli su navedeni brojevi od 1 do 5, ali nigde ne piše šta oni zapravo predstavljaju. \odgovor{Dodat je pasus iznad tabele koji objašnjava vrednosti datih brojeva.}

\section{Provera sadržajnosti i forme seminarskog rada}
% Oдговорите на следећа питања --- уз сваки одговор дати и образложење

\begin{enumerate}
\item Da li rad dobro odgovara na zadatu temu?\\
Da, pre nego što sam počeo da čitam, pretpostavio sam o čemu će biti reč, upravo to je bio sadržaj ovog rada.

\item Da li je nešto važno propušteno?\\
Sem stvari koje su navedene u odeljku 2, ne.

\item Da li ima suštinskih grešaka i propusta?\\
Ne, smatram da je suština ove problematike dobro opisana ovim radom.

\item Da li je naslov rada dobro izabran?\\
Da, po mom mišljenju naslov rada jednoznačno određuje sadržaj rada.

\item Da li sažetak sadrži prave podatke o radu?\\
Da, sažetak je u kratkim crtama opisao podatke o radu.

\item Da li je rad lak-težak za čitanje?\\
Nije ni lak ni težak, srednji nivo.

\item Da li je za razumevanje teksta potrebno predznanje i u kolikoj meri?\\
Predznanje ne, smatram da je potrebno iskustvo studiranja, bar jedna godina.

\item Da li je u radu navedena odgovarajuća literatura?\\
Da.

\item Da li su u radu reference korektno navedene?\\
Da.

\item Da li je struktura rada adekvatna?\\
Prilično sam siguran da jeste.

\item Da li rad sadrži sve elemente propisane uslovom seminarskog rada (slike, tabele, broj strana...)?\\
Da.

\item Da li su slike i tabele funkcionalne i adekvatne?\\
Da.
\end{enumerate}

\section{Ocenite sebe}
% Napišite koliko ste upućeni u oblast koju recenzirate: 
% a) ekspert u datoj oblasti
% b) veoma upućeni u oblast
% c) srednje upućeni
% d) malo upućeni 
% e) skoro neupućeni
% f) potpuno neupućeni
% Obrazložite svoju odluku
Smatram za sebe da sam veoma upućen u oblast, jer studiram duži niz godina i susreo sam se sa svim izazovima koji su opisani ovim radom.


\chapter{Dodatne izmene}
%Ovde navedite ukoliko ima izmena koje ste uradili a koje vam recenzenti nisu tražili. 

\end{document}
